Tentukan semua tripel bilangan asli $(a,b,c)$ sehingga ketiga bilangan berikut:
\begin{align*}
    2^a+2^b+3(c+1), \hspace{10pt} 2^b+2^c+3(a+1), \hspace{10pt} 2^c+2^a+3(b+1)
\end{align*}
semuanya adalah perpangkatan dua (dengan kata lain, dalam bentuk $2^k$ untuk suatu bilangan asli $k$).
\begin{solusi}
    Dengan induksi mudah dibuktikan untuk sembarang bilangan asli $x \ge 3$  berlaku $3(x+1) < 2^{x+1}$ dan ubtuk sembarang bilangan asli $x \ge 7$ berlaku $3(x+1) < 2^{x-1}$. Perhatikan untuk karena ketiga bilangan tersebut bernilai lebih dari 2 dan genap, maka haruslah $a,b,c$ ganjil agar $3(a+1),3(b+1),3(c+1)$ genap.
    WLOG $a \le b \le c$. Akan dibagi kasus berdasarkan hubungan $b$ dan $c$.

    \begin{itemize}
        \item $b = c$.\\
        Untuk $c \ge 3$ berlaku
        \begin{align*}
            2^{c+1} = 2^c + 2^c < 2^b + 2^c + 3(a+1) < 2^c + 2^c +2^{c+1} = 2^{c+2}
        \end{align*}
        yang menunjukkan bahwa $2^b + 2^c + 3(a+1)$ bukan berbentuk 2 pangkat untuk $c \ge 3$. Namun, untuk $c=1$ dari keadaan $a \le b \le c$ memaksa $a=b=c=1$ yang membuat $2^b+2^c+3(a+1)=10$ yang juga bukan berbentuk 2 pangkat.

        \item $b < c$.\\
        Tinjau $b \ge 3$.
        \begin{itemize}
        \item Jika $a=b=3$, maka $2^3+2^c+3(3+1)=2^c+20>16$ merupakan bilangan 2 pangkat. Perhatikan $c>3$ menyebabkan $2^c+20 \equiv 4 \mod 16$. Padahal karena bilangan tersebut adalah 2 pangkat yang lebih dari 16 haruslah l, bilangan tersebut haruslah bernilai 0 modulo 16. Kontradiksi.

        \item Jika $a=b > 3$ atau $b > a \ge 3$ maka $c \ge 7$ (ingat mereka harus ganjil). Maka
        \begin{align*}
            2^{c} &< 2^b + 2^c + 3(a+1)
            < 2^b + 2^c + 2^{a-1} \le 2^{c-1} + 2^c + 2^{c-1} = 2^c + 2^{c} = 2^{c+1}
        \end{align*}
        yang menunjukkan $2^b + 2^c + 3(a+1)$ bukan berbentuk 2 pangkat.
        \end{itemize}
      
        Dari kasus tersebut, $b \ge 3$ tidak memenuhi. Oleh karena itu, haruslah $1 = b \ge a$ atau $a=b=1$ sehingga $2^c+2^a+3(b+1)=2^c+8$. Agar $2^c+8$ berbentuk 2 pangkat, haruslah $2^c=8 \implies c=3$. Di cek, solusi $(1,1,3)$ memenuhi.
    \end{itemize}
    Dari sini didapat bahwa solusi yang memenuhi hanyalah $(1,1,3), (1,3,1), (3,1,1)$.
\end{solusi}