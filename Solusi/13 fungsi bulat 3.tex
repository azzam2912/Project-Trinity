Tentukan semua fungsi $f : \mathbb{N} \to \mathbb{N}$ yang memenuhi:
\[
f(a)+f(b) \mid a+b, \forall a,b \in \mathbb{N}
\]
\begin{solusi}
    Misal $P(a,b)$ adalah asersi ekspresi di soal. $P(n,n)$ untuk sembarang $n \in \NN$ menghasilkan
    \begin{align*}
        f(n)+f(n) \mid n+n \implies f(n) \mid n.
    \end{align*}
    Berarti $f(1) \mid 1$ menyebabkan $f(1)=1$. $P(2,1)$ menghasilkan  
    $$f(2)+f(1) \mid 2+1 \implies f(2)+1 \mid 3 \implies f(2)=2.$$ Akan dibuktikan bahwa $f(n)=n$ untuk semua $n \in \NN$ dengan induksi di $n$.
    Perhatikan untuk kasus $n=1$ dan $n=2$ jelas benar. Andaikan untuk $n=k\ge 2$ benar bahwa $f(k)=k$. 
    
    Observasi untuk $n=k+1$. Asersi $P(k+1,k)$ menghasilkan
    \begin{align*}
        f(k+1)+f(k) \mid k+1 + k \implies f(k+1)+k \mid 2k+1.
    \end{align*}
    Di lain pihak kita juga punya $f(k+1) \mid k+1$. Oleh karena itu untuk suatu $c,d \in \NN$ dapat dimisalkan $2k+1 = c(f(k+1)+k)$ dan $ k+1 = df(k+1) $
    sehingga akan didapat
    \begin{align*}
        k + k+1 &= cf(k+1)+ck\\
        k + df(k+1) &= cf(k+1) + ck\\
        (d-c)f(k+1) &= (c-1)k.
    \end{align*}
    Karena $gcd(k+1,k) =1$ dan $f(k+1) \mid k+1$, maka $gcd(f(k+1),k)=1$ sehingga haruslah $f(k+1) \mid c-1$ atau setara dengan
    \begin{align*}
        f(k+1) &\mid \dfrac{2k+1}{f(k+1)+k} - 1\\
        f(k+1)(f(k+1)+k) &\mid 2k+1 - f(k+1) - k\\
        f(k+1)^2+kf(k+1) &\mid k+1 - f(k+1).
    \end{align*}
    Andaikan $f(k+1) \neq k+1$ maka $1 \le f(k+1) < k+1$ sehingga
    \begin{align*}
        f(k+1)^2 &\le (k+1)(1-f(k+1)) \le (k+1)0 = 0
    \end{align*}
    yang merupakan sebuah kontradiksi. Oleh karena itu haruslah $f(k+1)=k+1$.

    Dari induksi matematika, terbukti bahwa $f(n)=n$ untuk semua bilangan asli $n$.
\end{solusi}

\begin{solusi}[2] (Credit to Farabi Azzam)
    Akan dibuktikan dengan induksi kuat bahwa $f(n)=n$ untuk semua bilangan asli $n$. Dari solusi sebelumnya didapat $f(1)=1$, $f(2)=2$, $f(3)=3$, base case terbukti. Asumsikan untuk semua $k=1,2,\dots,m$ dengan $m \ge 3$ berlaku $f(k)=k$. Berdasarkan Bertrand's Postulate, untuk $k>1$ terdapat bilangan prima $p$ sehingga $k+2<p<2(k+2)$. Oleh karena itu
    \begin{align*}
        f(k+1)+f(p-k-1) \mid (k+1)+(p-k-1)
    \end{align*}
    karena $1\le p-k-1<k+1$ maka berdasarkan asumsi induksi kuat, $f(p-k-1)=p-k-1$ sehingga didapat
    \begin{align*}
        f(k+1)+p-k-1 \mid p.
    \end{align*}
    Karena $f(k+1)+p-k-1>1$ dan $p$ prima, haruslah $f(k+1)+p-k-1=p$ yang menyebabkan $f(k+1)=k+1$. Dapat disimpulkan dengan induksi kuat bahwa $f(n)=n$ untuk semua $n \in \NN$. \qed
\end{solusi}