(a) Show that the equation
\[\lfloor x \rfloor (x^2 + 1) = x^3,\]
where $\lfloor x \rfloor$ denotes the largest integer not larger than $x$, has exactly one real solution in each interval between consecutive positive integers.

(b) Show that none of the positive real solutions of this equation is rational.

(Baltic Way 2012 Problem 3)
\begin{solusi}
    Akan ditunjukkan bahwa ada tepat satu $x \in [n, n+1)$ untuk sembarang $n \in \NN$.
    Perhatikan jika $x$ bulat, maka
    \begin{align*}
        x(x^2+1) &= x^3\\
        x &= 0
    \end{align*}    
    yang kontradiktif dengan syarat $x \ge 1$. Oleh karena itu dapat dimisalkan $x=y+\epsilon$ dimana $y \in \NN$ dan $\epsilon \in (0,1)$. 
    Perhatikan fungsi $f:\RR\to\RR$ dengan $f(\epsilon) = \epsilon^3+2y\epsilon^2+y^2\epsilon-y$ untuk sembarang $y > 0$. Perhatikan bahwa fungsi ini bersifat monoton naik untuk $\epsilon \ge 0$ karena $f'(\epsilon) = 3\epsilon^2+4y\epsilon+y^2$.

    Karena $f(0) = -y < 0$ dan $f(1)=1+2y+y^2-y=1+y+y^2>0$ maka berdasarkan Intermediate Value Theorem ada $\epsilon \in (0,1)$ yang memenuhi $f(\epsilon)=0$. Karena sebelumnya diketahui $f$ monoton naik, maka ada tepat satu $\epsilon \in (0,1)$ yang memenuhi $f(\epsilon)=0$.

    Dari sini persamaan di soal menjadi menjadi
    \begin{align*}
        y((y+\epsilon)^2+1) &= (y+\epsilon)^3\\
        y^3+2y^2\epsilon+y\epsilon^2+y &= y^3+3y^2\epsilon+3y\epsilon^2+\epsilon^3\\
        \epsilon^3+2y\epsilon^2+y^2\epsilon-y &= 0\\
        f(\epsilon) &= 0
    \end{align*}
    yang telah terbukti sebelumnya, mempunyai tepat satu solusi $\epsilon \in (0,1)$. Bagian (a) terbukti.

    Sekarang, andaikan ada solusi $x=y+\epsilon$ rasional yang memenuhi soal. Oleh karena itu dapat diasumsikan ada $\epsilon=\frac{a}{b}$ dimana $a,b \in \NN$, $1 \le a<b$, dan $gcd(a,b)=1$ sehingga $f(\epsilon)=0$. Dengan substitutsi nilai $\epsilon = \frac{a}{b}$ ke $f(\epsilon)=0$ didapat
    \begin{align*}
        \left(\frac{a}{b}\right)^3+2y\left(\frac{a}{b}\right)^2+y^2\left(\frac{a}{b}\right)-y &= 0\\
        a^3+2ya^2b+y^2ab^2 &= yb^3
    \end{align*}
    lihat sisi kanan persamaan tersebut, $yb^3$, terbagi oleh $b$. Oleh karena itu $b \mid a^3+2ya^2b+y^2ab^2 \implies b \mid a^3$. Kontradiksi dengan syarat $gcd(a,b)=1$ dan $b>1$. 

    Dapat disimpulkan bahwa semua solusi $x$ dari persamaan tersebut irasional. Bagian (b) terbukti.
    
\end{solusi}