Misalkan $f(x) = 1 + \frac{90}{x}$. Nilai terbesar $x$ yang memenuhi
\[
f(f(\cdots f(x) \cdots)) = x,
\]
\text{(2019 kali aplikasi fungsi $f$)}
adalah \ldots
\begin{solusi}
Jika $f(x)=x$ maka 
\begin{align*}
    9+\dfrac{10}{x} &= x\\
    9x+10 &= x^2\\
    (x-10)(x+9) &= 0
\end{align*}
berarti $x$ terbesar adalah $x=10$. Sekarang, andaikan $x \neq 10$ dan $x > 0$. Jika $x>10$ maka
\begin{align*}	
f(x) = 9+\dfrac{10}{x} < 9 + \dfrac{10}{10} = 10.
\end{align*}
Jika $x<10$ maka
\begin{align*}
f(x) = 9+\dfrac{10}{x} > 9 + \dfrac{10}{10} = 10
\end{align*}
Oleh karena itu, jika $x>10$ maka $f^{2019}(x)<10<x$ dan jika $x<10$ maka $f^{2019}(x)>10>x$. Kasus saat $x \neq 9$ dan $x<0$ juga merupakan analogi cara tersebut.
Terbukti bahwa $x$ terbesar yang mungkin adalah $\boxed{10}$.
\end{solusi}