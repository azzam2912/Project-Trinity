% https://math.stackexchange.com/questions/481673/gxygxgy-gxygxgy-for-all-x-y
Find all functions $f:\RR \to \RR$ which satisfy
\begin{align*}
    f(x+y)+f(x)f(y) = f(x)+f(y)+f(xy)
\end{align*}
for all $x,y \in \RR$.
\begin{solusi}
    Let $P(x,y)$ be the assertion of the expression on the problem.\\
    $P(2,2)$ then $f(4)+f(2)^2 = 2f(2)+f(4) \implies f(2)^2-2f(2)=0 \implies f(2)=0 \vee f(2)=2$.\\
    $P(1,1)$ then $f(2)+f(1)^2 = 3f(1)$.
    \begin{itemize}
        \item If $f(2)=0$ then $f(1)^2-3f(1)=0 \implies f(1)=0 \vee f(1)=3$.
        \item If $f(2)=2$ then $f(1)^2-3f(1)+2=0 \implies f(1)=1 \vee f(1)=2$.
    \end{itemize}
    Hence there are several cases.
    \begin{enumerate}[I.]
        \item $f(2)=0 \wedge f(1)=0$\\
        $P(x,1)$ produce
        \begin{align*}
            f(x+1)+f(x)f(1) &= f(x)+f(1)+f(x)\\
            f(x+1) + 0 &= f(x) + 0 + f(x)\\
            f(x+1) &= 2f(x)\\
            \dots\\
            f(x+n) &= 2^nf(x)
        \end{align*}
        hence we have $f(x+4) = 2^4f(x) = 16f(x)$ and $f(4)=f(2+2)=2^2f(2)=0$. Now
        $P(x,2)$ produces
        \begin{align*}
            f(x+2)+f(x)f(2) &= f(x)+f(2)+f(2x)\\
            f(x+2) + 0 &= f(x) + 0 + f(2x)\\
            4f(x) &= f(x)+f(2x)\\
            f(2x) &= 3f(x)
        \end{align*}
        hence $f(4x) = 3f(2x)=9f(x)$. Thus by those results, $P(x,4)$ will produces
        \begin{align*}
            f(x+4)+f(x)f(4) &= f(x)+f(4)+f(4x)\\
            16f(x) + f(x) \cdot 0 &= f(x) + 0 + 9f(x)\\
            7f(x) &= 0\\
            f(x) &= 0.
        \end{align*}
        Therefore this case $f(x)=0$ satisfy for all real $x$.
        
        \item $f(2)=0 \wedge f(1)=3$\\
        $P(x,1)$ produces
        \begin{align*}
            f(x+1)+f(x)f(1) &= f(x)+f(1)+f(x)\\
            f(x+1)+3f(x) &= f(x)+3+f(x)\\
            f(x+1) &= 3-f(x)
        \end{align*}
        therefore 
        \begin{align*}
            f(x+2) = 2 - f(x+1) = 3-(3-f(x)) = f(x).
        \end{align*}
        Hence $P(\frac{1}{2}, 2)$ produces
        \begin{align*}
            f(\frac{1}{2}+2) + f(\frac{1}{2})f(2) &= f(\frac{1}{2})+f(2)+f(\frac{1}{2}\times2)\\
            f(\frac{1}{2}) + 0 &= f(\frac{1}{2}) + 0 +f(1)\\
            f(1) &= 0
        \end{align*}
        which is a contradiction. No satisfying answers for this case.
        
        \item $f(2)=2 \wedge f(1)=1$\\
        $P(x,1)$ produces
        \begin{align*}
            f(x+1)+f(x)f(1) &= f(x)+f(1)+f(x)\\
            f(x+1)+f(x) &= f(x)+1+f(x)\\
            f(x+1) &= f(x) + 1
        \end{align*}
        and it is easy to show that $f(x+n)=f(x)+n$ and $f(n)=f(1+n-1)=f(1)+n-1=n$ for any positive integers $n$.

        Now $P(x,n)$ for arbitrary positive integer $n$ will gives
        \begin{align*}
            f(x+n)+f(x)f(n) &= f(x)+f(n)+f(xn)\\
            f(x+n)+nf(x) &= f(x)+n+f(nx)\\
            f(x)+n+nf(x) &= f(x)+n+f(nx)\\
            f(nx) &= nf(x)
        \end{align*}

        Next, $P(2x,2y)$ produces
        \begin{align*}
            f(2x+2y)+f(2x)f(2y) &= f(2x)+f(2y)+f(2x2y)\\
            2f(x+y)+2f(x)2f(y) &= 2f(x)+2f(y)+4f(xy)\\
            f(x+y)+2f(x)f(y) &= f(x)+f(y)+2f(xy)
        \end{align*}

        subtract the last equation with the equation in the problem statement, we get for any real numbers $x,y$ that $f(x)f(y)=f(xy)$ hence $f(x+y)=f(x)+f(y)$. Those equations are indeed Cauchy's Functional Equation where $f(x)=x$. This case finished.


        \item $f(2)=2 \wedge f(1)=2$\\
        $P(x-1,1)$ gives
        \begin{align*}
            f((x-1)+1) + f(x-1)f(1) &= f(x-1)+f(1)+f((x-1)1)\\
            f(x) + 2f(x-1) &= f(x-1)+2+f(x-1)\\
            f(x) &= 2.
        \end{align*}
        Thus $f(x)=2$ for all real number $x$.
    \end{enumerate}
    By considering all cases, we get the satisfying functions for all real number $x$ : $f(x)=0$, $f(x)=2$, or $f(x)=x$.
\end{solusi}