\textbf{English}
Let $a$, $b$ and $c$ be real numbers such that $a + b + c = 8$ and $ab + bc + ca = 0$.
Find the maximum value of $3(a + b)$.

\textbf{Bahasa Indonesia}
Misalkan $a$, $b$ dan $c$ adalah bilangan real yang memenuhi $a + b + c = 8$ dan $ab + bc + ca = 0$.
Tentukan nilai maksimum dari $3(a + b)$.


\begin{solusi}
\textbf{Solution (English)}
Note that $a+b = 8-c$. Hence
\begin{align*}
    ab &= -bc-ca\\
    ab &= -c(b+a)\\
    ab &= -c(8-c)
\end{align*}
Using Vieta's Theorem, consider a quadratic equation with roots $a$ and $b$:
\begin{align*}
    x^2 - (a+b)x + ab &= 0\\
    x^2 - (8-c) - c(8-c) &= 0
\end{align*}
Since $a,b,c$ are real numbers, the discriminant of this quadratic equation must be nonnegative:
\begin{align*}
    (-(8-c))^2 - 4(1)(-c(8-c)) &\ge 0\\
    (8-c)(8-c) + (8-c)(4c) &\ge 0\\
    (8-c)(8-c+4c) &\ge 0\\
    (c-8)(3c+8) &\le 0
\end{align*}
which shows that $-\frac{8}{3} \le c \le 8$.
From here we get $3(a+b) = 3(8-c) \le 3(8-(-\frac{8}{3})) = 32$.
$\therefore$ the maximum value of $3(a + b)$ is 32, which is achieved when $c=-\frac{8}{3}$.

\textbf{Solusi (Bahasa Indonesia)}
Perhatikan bahwa $a+b = 8-c$ sehingga
\begin{align*}
    ab &= -bc-ca\\
    ab &= -c(b+a)\\
    ab &= -c(8-c)
\end{align*}
Dengan Teorema Vieta, dapat ditinjau persamaan kuadrat dengan akar-akar $a$ dan $b$:
\begin{align*}
    x^2 - (a+b)x + ab &= 0\\
    x^2 - (8-c) - c(8-c) &= 0
\end{align*}
Karena $a,b,c$ real, maka diskriminan persamaan kuadrat tersebut nonnegatif:
\begin{align*}
    (-(8-c))^2 - 4(1)(-c(8-c)) &\ge 0\\
    (8-c)(8-c) + (8-c)(4c) &\ge 0\\
    (8-c)(8-c+4c) &\ge 0\\
    (c-8)(3c+8) &\le 0
\end{align*}
yang menunjukkan $-\frac{8}{3} \le c \le 8$.
Dari sini didapat $3(a+b) = 3(8-c) \le 3(8-(-\frac{8}{3})) = 32$.

$\therefore$ Nilai maksimum dari $3(a + b)$ adalah 32 yang tercapai saat $c=-\frac{8}{3}$.
\end{solusi}