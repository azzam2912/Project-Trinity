Misalkan $P(x)$ dan $Q(x)$ adalah polinomial tak konstan dengan koefisien-koefisien real. Buktikan bahwa jika
\[ \lfloor P(y) \rfloor = \lfloor Q(y) \rfloor \]
untuk semua bilangan real $y$, maka $P(x) = Q(x)$ untuk semua bilangan real $x$.

\begin{solusi}
Akan ditunjukkan bahwa $P(x)=Q(x)$ untuk semua $x \in \mathbb{R}$.

\begin{claim*}
    $|P(x) - Q(x)| < 1$ untuk semua $x \in \mathbb{R}$.
    \begin{bukti}
        Kita akan buktikan dengan kontradiksi. Andaikan, WLOG, terdapat suatu $y_0$ sehingga $P(y_0) \ge Q(y_0) + 1$. Maka, dengan mengambil fungsi floor di kedua sisi, kita peroleh:
        \begin{align*}
            \lfloor P(y_0) \rfloor \ge \lfloor Q(y_0) + 1 = \lfloor Q(y_0) \rfloor + 1 > \floor{Q(y_0)}
        \end{align*}
        Hal ini kontradiksi dengan hipotesis awal bahwa $\lfloor P(y) \rfloor = \lfloor Q(y) \rfloor$ untuk semua bilangan real $y$. Dengan argumen yang sama untuk kasus $Q(y_0) \ge P(y_0) + 1$, kita dapat menyimpulkan bahwa tidak mungkin $|P(x) - Q(x)| \ge 1$. Jadi, haruslah $|P(x) - Q(x)| < 1$ untuk semua $x \in \mathbb{R}$.
    \end{bukti}
\end{claim*}


Misalkan $R(x) = P(x) - Q(x)$. Karena $P(x)$ dan $Q(x)$ adalah polinomial, maka $R(x)$ juga merupakan polinomial. Dari klaim sebelumnya, kita tahu bahwa $|R(x)| < 1$ untuk semua $x \in \mathbb{R}$. Perhatikan bahwa polinomial yang terbatas (bounded) di seluruh domainnya haruslah merupakan polinomial konstan. Jadi,
\[ P(x) - Q(x) = C \]
untuk suatu konstanta real $C$, dengan $|C| < 1$.

Sekarang akan dibuktikan bahwa $C=0$ dengan kontradiksi. Andaikan $C \neq 0$. Tanpa mengurangi keumuman, asumsikan $C>0$. \\
Karena $Q(x)$ adalah polinomial tak konstan, rangenya adalah $(-\infty, \infty)$. Karena $Q(x)$ kontinu, berdasarkan Teorema Nilai Antara, kita dapat menemukan suatu $x_0 \in \mathbb{R}$ sehingga $Q(x_0) = k - C$ untuk suatu bilangan bulat $k$.

Sekarang kita periksa nilai floor dari $P(x_0)$ dan $Q(x_0)$:
\[ \lfloor P(x_0) \rfloor = \lfloor Q(x_0) + C \rfloor = \lfloor (k-C) + C \rfloor = \lfloor k \rfloor = k \]
\[ \lfloor Q(x_0) \rfloor = \lfloor k-C \rfloor \]
Karena kita mengasumsikan $C>0$, maka $k-C < k$. Akibatnya, $\lfloor k-C \rfloor \le k-1 < k$.
Ini menunjukkan bahwa $\lfloor Q(x_0) \rfloor < \lfloor P(x_0) \rfloor$, yang sekali lagi bertentangan dengan hipotesis awal, didapat haruslah $C=0$. Karena $P(x) - Q(x) = C$ dan $C=0$, maka $P(x)-Q(x)=0$, yang berarti $P(x)=Q(x)$ untuk semua bilangan real $x$. Terbukti.

\end{solusi}