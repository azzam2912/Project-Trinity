Find all functions $f:\mathbb{R}\rightarrow\mathbb{R}$ that satisfy
\[
f(x^2-y)+2yf(x)=f(f(x))+f(y)
\]for all $x,y\in\mathbb{R}$.

(USAJMO 2024 Problem 5, Proposed by Carl Schildkraut)
\begin{solusi}
    Misalkan $P(x,y)$ asersi pada fungsi di soal.\\
    $P(0,0)$ menghasilkan
    \begin{align*}
        f(0)+0 &= f(f(0))+f(0)\\
        f(f(0)) &= 0
    \end{align*}
    $P(f(0),0)$ mengakibatkan
    \begin{align*}
        f(f(0)^2)+0 &= f(f(f(0))) + f(0)\\
        f(f(0)^2) &= f(0) + f(0)\\
        f(f(0)^2) &=2f(0).
    \end{align*}
    $P(f(0),f(0)^2)$ menyebabkan
    \begin{align*}
        f(0)+2f(0)^2f(f(0)) &= f(f(f(0)))+f(f(0)^2)\\
        f(0)+2f(0)^2 \cdot 0 &= f(0)+2f(0)\\
        f(0) &= 0
    \end{align*}
    Maka $P(x,0)$ menyebabkan
    \begin{align*}
        f(x^2)+0 &= f(f(x)) + f(0)\\
        f(x^2) &= f(f(x))
    \end{align*}
    Selanjutnya $P(x,x^2)$ mengakibatkan
    \begin{align*}
        f(0) + 2x^2f(x) &= f(f(x))+f(x^2)\\
        0+2x^2f(x) &= f(x^2)+f(x^2)\\
        x^2f(x) &= f(x^2) = f(f(x))
    \end{align*}
    sehingga $f(f(0))=f(0)=0$.
    Di lain sisi $P(0,y)$ membuat
    \begin{align*}
        f(-y)+2yf(0) &= f(f(0)) + f(y)\\
        f(-y) &= f(y)
    \end{align*}
    yang menunjukkan $f$ adalah fungsi genap. Sekarang perhatikan untuk $x,y > 0$, asersi $P(\sqrt{x}, y)$ dan $P(\sqrt{y},x)$ mengakibatkan
    \begin{align*}
        f(x-y) + 2yf(\sqrt{x}) &= f(f(\sqrt{x}))+f(y)\\
        f(x-y) + 2yf(\sqrt{x}) &= f(x)+f(y)
    \end{align*}
    dan
    \begin{align*}
        f(y-x) + 2xf(\sqrt{y}) &= f(f(\sqrt{y}))+f(x)\\
        f(y-x) + 2xf(\sqrt{y}) &= f(y)+f(x).
    \end{align*}
    Karena $f$ fungsi genap, maka $f(x-y)=f(y-x)$ sehingga didapat
    \begin{align*}
        2yf(\sqrt{x}) &= 2xf(\sqrt{y})\\
        f(\sqrt{x}) &= \dfrac{f(\sqrt{y})}{y}\cdot x.
    \end{align*}
    Jika $y$ dibuat menjadi fix point dan $x$ sembarang, maka $\dfrac{f(\sqrt{y})}{y}=c$ adalah suatu konstanta real sehingga $f(\sqrt{x})=cx$ atau $f(x)=cx^2$. Oleh karena itu, substitusi kembali $P(x,y)$ pada soal akan menghasilkan
    \begin{align*}
        f(x^2-y)+2yf(x)&=f(f(x))+f(y)\\
        c(x^2-y)^2+2ycx^2 &= f(cx^2) + cy^2\\
        cx^4-2cx^2y+cy^2 + 2cx^2y &= c\cdot c(cx^2)^2 + cy^2\\
        cx^4 &= c^3x^4\\
        cx^4(c-1)(c+1)&=0\\
        c &= -1,0,1
    \end{align*}
    Cek kembali ke soal, didapatkan fungsi $f$ yang memenuhi adalah
    \begin{align*}
        f(x) = 
        \begin{cases}
            -x^2\\
            0\\
            x^2
        \end{cases}
    \end{align*}
     untuk sembarang $x \in \RR$
\end{solusi}